\thispagestyle{plain}			% Supress header 
% \setlength{\parskip}{0pt plus 1.0pt}
\section*{Abstract}

The COVID-19 pandemic has profoundly impacted global supply chains, particularly in the sectors of e-commerce and perishable goods. This study investigates the significant disruptions experienced by suppliers of perishable goods and the subsequent strategic adaptations made to enhance supply chain resilience. The pandemic catalyzed a rapid transformation in consumer behavior, leading to a surge in online shopping and an increased reliance on digital platforms. This shift exacerbated supply chain vulnerabilities, particularly during the initial stages of the pandemic, where inventory shortages, transportation delays, and logistical bottlenecks were prevalent. The challenges were further intensified by external events such as the Suez Canal blockage, which underscored the fragility of global supply chains and the necessity for rapid adjustments in inventory management strategies.

The study also highlights the contrasting effects of the pandemic across different regions, with significant disruptions observed in both the U.S. and Asia. In the U.S., e-commerce businesses faced additional fulfillment challenges due to lockdowns and restrictions, particularly in the distribution of non-essential products. In Asia, the pandemic led to factory shutdowns, causing delays in manufacturing and a shortage of critical items, particularly in the healthcare, automotive, and food industries. These disruptions were compounded by restrictions on movement, leading to delays and cancellations of shipments and further bottlenecks in cross-border trade.

In response to these challenges, suppliers of perishable goods have implemented a range of strategies to enhance supply chain resilience. These include local warehousing, diversification of suppliers, and increased collaboration across the supply chain. The adoption of technology for real-time data sharing and the creation of crisis management teams are indicative of a proactive approach to managing future disruptions. The evidence suggests a strategic shift towards more resilient and flexible supply chains, with a focus on improving robustness and preparedness in anticipation of future crises.

With a mixed-method approach of questionnaires and interviews we discuss the significant disruptions in the supply chain of perishable goods due to COVID-19 and provide methods and strategies the suppliers and supermarkets have used to minimize the impact on the supply chain and increase the resilience of it to future similar events. The study concludes that the pandemic has led to substantial changes in e-commerce strategies and logistics operations, emphasizing the need for continued adaptation to the evolving market and consumer needs in the post-pandemic era.











% KEYWORDS (MAXIMUM 10 WORDS)
\vfill
\begin{flushleft}
Keywords: Supply Chain, Disruption, Resilient Supply Chain, Digitalization, Long-term Strategy.
\end{flushleft}
\break
% References: \parencite{fryer_2020_understanding, zhao_2018_research, Musella2023TheFragilities}

% \newpage				% Create empty back of side
% \thispagestyle{empty}
% \mbox{}