\chapter{Methodology}

This study was initially conceived as a qualitative investigation into the resilience and challenges of supply chains during the COVID-19 pandemic. Our primary intention was to conduct an in-depth qualitative analysis, focusing on gathering detailed, narrative data. However, as we delved into relevant literature and reviewed established methodologies, it became evident that a mixed-methods approach would be more effective in addressing the complexity of the research questions. Consequently, we developed a survey that included both qualitative and quantitative elements, providing a broader range of data to inform our analysis. To complement the survey findings, a personal interview with a key participant in the supply chain sector was conducted, enriching the qualitative aspect of our study.

To further refine and explore our research questions (RQs), we considered various methodological approaches, guided by the insights of \textcite{Ghauri2020ResearchStudies}. Ultimately, we designed a comprehensive questionnaire incorporating a mix of Likert-scale questions \parencite{Joshi2015LikertExplained,Batterton2017MilitaryIt,Mirahmadizadeh2018DesigningData} and yes/no questions to test our hypotheses. Additionally, we included a free-text section in the survey, encouraging respondents to provide detailed recommendations and observations beyond the structured questions. This open-ended question proved invaluable for the qualitative analysis, offering nuanced insights that supported our overall findings. The combination of these approaches allowed for an investigation into supply chain resilience, blending quantitative aspect with qualitative depth.

\section{The Survey}

\subsection{Format of the survey}
In designing our survey for this study, we utilized a combination of Likert scale questions, yes/no questions, and a single subjective question, as previously mentioned. The Likert scale, a fundamental tool in social science research, was particularly useful for quantifying attitudes, beliefs, and behaviors related to supply chain resilience during the COVID-19 pandemic. By converting qualitative perceptions into quantitative data, it allowed us to perform a robust statistical analysis of these subjective constructs. While Likert scales can vary in length, we chose a 5-point scale (ranging from Strongly Agree to Strongly Disagree) to simplify the process for our respondents, thereby reducing their burden and encouraging higher completion rates. Our approach involved careful consideration of the number and type of questions included in the survey. Although it is common to use extensive Likert-scale questionnaires for detailed statistical analysis, we opted to limit the number of these questions to three, recognizing the low response rate and the need to maintain participant engagement. This decision was influenced by the initial literature review, which indicated that reducing the complexity of the survey could lead to more reliable data collection. The yes/no questions were strategically included to provide clear, straightforward insights into whether companies had implemented measures to address specific supply chain challenges, without delving into the success or failure of those measures. To develop the Likert scale questions, we conducted a thorough review of current topics in supply chain management, with a focus on resilience and challenges during the pandemic, as highlighted in several key publications. These questions were designed to be clear, concise, and free from bias, each aiming to capture the respondents' level of agreement or disagreement with statements reflecting critical aspects of their internal processes. The layout of our survey was meticulously crafted to minimize confusion and bias. We ensured that instructions were clear and emphasized that there were no right or wrong answers, which was crucial for maintaining the anonymity and confidentiality of respondents. This approach was intended to foster honest and thoughtful responses. The inclusion of a single open-ended question at the end of the survey allowed participants to provide additional insights or recommendations, adding a qualitative dimension to our analysis. In terms of administration, the survey was conducted online to maximize reach and convenience, given the constraints of the pandemic. This method was chosen to increase the response rate and ensure that we collected a representative sample. The combination of Likert scale questions with yes/no questions provided a comprehensive view of supply chain resilience, allowing us to understand not only the existence of specific measures but also the attitudes and thought processes behind them. This mixed-method approach ultimately provided a nuanced understanding of the challenges faced by supply chains during the pandemic.

\subsection{Survey Design and Rationale}

The survey developed for this study was meticulously designed to capture a comprehensive understanding of the impact of the COVID-19 pandemic on supply chain resilience, particularly within the context of perishable goods. Each question was carefully crafted to elicit specific insights while allowing for future in-depth analysis based on the data collected. The rationale behind the selection and formulation of each question reflects our intent to gather both quantitative and qualitative data that can inform broader discussions on supply chain strategies and their efficacy during unprecedented disruptions.

\subsubsection*{Identification of Perishable Goods}
The initial question aimed to categorize the type of perishable goods managed by the respondents' companies. Recognizing the diversity in perishable goods—ranging from vegetables and dairy products to meat, poultry, and seafood—this question was essential for contextualizing the subsequent responses. By allowing participants to select multiple categories, the survey accounted for companies dealing in various types of perishable goods. This categorization facilitated the understanding of industry-specific challenges and also enabled potential comparative analysis across different types of perishable goods. Defining "perishable goods" within the survey ensured that all participants had a consistent understanding, thereby reducing any ambiguity in their responses.

\subsubsection*{Company Turnover and Size}
The subsequent questions concerning the company’s turnover and size were designed to explore the potential correlation between a company’s scale and its response to supply chain disruptions. The turnover question was included to differentiate between small, medium, and large enterprises, allowing us to assess whether financial resources influenced the effectiveness of supply chain strategies during the pandemic. Larger companies, with their broader reach and potentially more complex supply chains, might have different strategies compared to smaller firms with more localized operations. This question set the stage for analyzing whether the scale of operations impacted a company’s ability to adapt to the pandemic-induced challenges. Similarly, the question regarding company size, classified into small (up to 50 employees), medium (51-250 employees), and large enterprises (over 250 employees), was intended to further refine our understanding of how organizational structure and capacity might influence supply chain resilience. By capturing this data, we anticipated the ability to segment responses and explore trends that might differ based on the size of the organization. For instance, larger enterprises might have more formalized strategies and resources, while smaller companies might rely on more agile and ad-hoc approaches.

\subsubsection*{Supply Chain Disruptions and Strategic Responses}
A critical component of the survey involved assessing the extent to which the COVID-19 pandemic disrupted supply chains, specifically for perishable goods. This question utilized a Likert scale, ranging from 1 (not at all) to 5 (very significantly), to capture the severity of the disruption as perceived by the respondents. The use of a Likert scale in this context allowed for a nuanced understanding of the impact, enabling respondents to express varying degrees of disruption. This data is crucial for understanding the scale of the challenge faced by different companies and sets the foundation for analyzing how these disruptions influenced subsequent strategic decisions. Following this, the survey inquired whether the company had implemented any changes to its supply chain strategies during the pandemic. This yes/no question aimed to capture a binary response, providing a clear indicator of whether companies actively adapted their supply chain practices in response to the crisis. This question was followed by another yes/no question regarding whether companies had diversified their supplier base—a common strategy to mitigate risks associated with supply chain disruptions. The sequential nature of these questions was intentional, designed to lead respondents from the recognition of disruption to specific strategic responses, thus creating a logical flow in the survey.

\subsubsection*{Evaluating the Effectiveness of Resilience Strategies}
The survey then probed the effectiveness of the resilience strategies that were in place before the pandemic. Using a Likert scale from 1 (not effective) to 5 (extremely effective), this question sought to gauge how well-prepared companies were to handle the unprecedented challenges posed by the pandemic. By asking participants to evaluate the success of their existing strategies, the survey aimed to identify potential gaps in preparedness and areas where companies may have over or underestimated their resilience. Further, the survey explored whether companies had developed any backup plans for future disruptions. This question, also structured as a yes/no option, was critical in understanding the lessons learned by companies during the pandemic and whether they had proactively taken steps to bolster their supply chain resilience against future crises. The inclusion of this question allowed us to assess the forward-thinking approaches of companies and their commitment to improving resilience post-pandemic.

\subsubsection*{Adjustments in Demand Forecasting and Pandemic Preparedness}
The survey also addressed whether companies had adjusted their demand forecasting methods in response to the pandemic. Accurate demand forecasting is pivotal in supply chain management, especially in times of uncertainty. The question sought to determine if companies had recognized the need for more adaptive forecasting techniques as market conditions rapidly changed during the pandemic. The final structured question asked whether the company had established a formal pandemic preparedness plan for future events. This yes/no question aimed to determine whether the experiences of the COVID-19 pandemic had led to the institutionalization of specific preparedness measures. The presence of such a plan would indicate a company’s strategic shift towards long-term resilience and readiness for potential future disruptions.

\subsubsection*{Subjective Insights from Participants}
To complement the structured questions, the survey included a subjective, open-ended question inviting respondents to share additional insights based on their personal experience of handling supply chain issues during the pandemic. This question was strategically placed at the end of the survey to provide participants with the opportunity to elaborate on any points not covered by the structured questions. It allowed for the collection of qualitative data that could offer deeper insights into the challenges and strategies employed by different companies. This open-ended format included to capture the unique perspectives and innovative solutions that may not have been addressed through the more structured questions.

\subsection{Survey Distribution and Targeting}

The survey for this study was created using Google Sheets, allowing us to generate a shareable link that could be easily distributed to potential participants. The complete set of survey questions can be found in Appendix \ref{appendix:survey}. The focus of our work was narrowed to contain only European countries as we, the authors, are located in European Union member countries and found that this demographic affects us the most. Our focus was exclusively on participants from Europe, reflecting our research's geographic concentration. To identify suitable participants, we conducted extensive research on LinkedIn. We filtered potential contacts by country and then searched for individuals holding prominent positions relevant to supply chain management, such as Purchasing Assistant, Store Manager, Location Manager, and Director of Procurement, among others. This approach enabled us to target professionals directly involved in supply chain operations across various European countries. The companies selected were among the top 5 chain supermarkets in countries with the highest GDP in the Union. Some of the countries selected are such but not limited to Sweden, Norway, Czech Republic, the Netherlands, England, Spain, Germany, and etc. This ensured that we take samples of countries with different Geo-locations. As can be seen, some countries are Nordic, which are larger landmass countries with scarcely populated cities, where other countries such as the Netherlands, Spain, and France are evenly populated or almost densely populated areas. After identifying potential participants on LinkedIn, we initiated contact primarily through LinkedIn messaging. In cases where contact details were not readily available, we supplemented our efforts by sending emails. For those whose emails were not publicly available, we employed a common email format (first name dot last name at the company's domain) to reach out to them. We ensured that participants were informed that their responses would remain anonymous, with no personal details being recorded. This information was clearly communicated in both the LinkedIn messages and emails sent to potential respondents. Additionally, we emphasized that participation in the survey was entirely voluntary, with no mandatory questions. This approach was designed to make it easier for participants to complete the survey, even if they chose not to answer every question. Understanding that respondents might be reluctant to spend significant time on surveys, we streamlined the process by designing the entire survey to fit on a single page, eliminating the need for a "next" button. Despite the absence of incentives such as prize money or other benefits, we encouraged participation by emphasizing the importance of their input in our research. In total, we targeted participants from 16 European countries and successfully contacted approximately 275 individuals across a range of supermarkets, including Netto, Coop, Lidl, Kiwi, Aldi, Carrefour, and more. For a detailed breakdown of the number of people contacted, the countries involved, the supermarkets represented, and the typical positions targeted, please refer to Table \ref{tab:survey-participants}.



\begin{table}[]
\scriptsize
\centering
\begin{tabular}{@{}|l|l|l|l|@{}}
\toprule
\textbf{Country} & \textbf{\begin{tabular}[c]{@{}l@{}}Number of \\ People \\ Contacted\end{tabular}} & \textbf{Supermarkets}                                                                                          & \textbf{Positions}                                                                                                                                                                                             \\ \midrule
Sweden           & 43                                                                                & \begin{tabular}[c]{@{}l@{}}Citigross, Martin \& Cervera, \\ Coop, Hemshop, IKA, Willys, \\ Mathem\end{tabular} & \begin{tabular}[c]{@{}l@{}}Store Manager, Purchasing Assistant, \\ Location Manager, Purchasing \\ Manager, Perishable Purchaser\end{tabular}                                                                  \\ \midrule
Norway           & 9                                                                                 & Kiwi, Coop                                                                                                     & \begin{tabular}[c]{@{}l@{}}Store Manager, Purchaser, Head of \\ Indirect Procurement, Senior \\ Strategic Buyer\end{tabular}                                                                                   \\ \midrule
Denmark          & 19                                                                                & Netto, Coop, Lidl                                                                                              & \begin{tabular}[c]{@{}l@{}}Senior Buyer, Category Manager, \\ Purchasing Assistant, \\ Category Director\end{tabular}                                                                                          \\ \midrule
Finland          & 4                                                                                 & S-Market, K-Supermarket, Lidl                                                                                  & \begin{tabular}[c]{@{}l@{}}Junior Purchasing Manager, \\ Purchasing and Sales Director\end{tabular}                                                                                                            \\ \midrule
Belgium          & 19                                                                                & Aldi, Carrefour, Lidl, Colruyt                                                                                 & \begin{tabular}[c]{@{}l@{}}Senior Buyer, Head of Sales \\ and Purchasing, Technical \\ Purchasing Manager, Reverse \\ Logistics Manager\end{tabular}                                                           \\ \midrule
Netherlands      & 15                                                                                & Albert Heijen, Aldi, Lidl                                                                                      & \begin{tabular}[c]{@{}l@{}}Manager Procurement, \\ Purchasing Manager, \\ Senior Director Procurement\end{tabular}                                                                                             \\ \midrule
Spain            & 34                                                                                & \begin{tabular}[c]{@{}l@{}}Marchedona, Carrefour, Lidl, \\ Aldi, Supercore\end{tabular}                        & \begin{tabular}[c]{@{}l@{}}CSR Manager, IT Procurement \\ Manager, Senior Buyer, Indirect \\ Purchasing VP, \\ Supply Chain Manager\end{tabular}                                                               \\ \midrule
Portugal         & 18                                                                                & \begin{tabular}[c]{@{}l@{}}Pingo Doce, Continente, \\ Intermarche\end{tabular}                                 & \begin{tabular}[c]{@{}l@{}}Category Manager, National Retail \\ Operations Director, Store Manager, \\ Digital Channels Director\end{tabular}                                                                  \\ \midrule
France           & 18                                                                                & \begin{tabular}[c]{@{}l@{}}Carrefour, Intermarche, \\ Monoprix\end{tabular}                                    & \begin{tabular}[c]{@{}l@{}}Global Category Management, \\ International Head of Buying, \\ Import Director, Director of \\ Supply Chain, Trader\end{tabular}                                                   \\ \midrule
Greece           & 9                                                                                 & \begin{tabular}[c]{@{}l@{}}Sclavenitis, Alpha Beta \\ Vassilopoulos\end{tabular}                               & \begin{tabular}[c]{@{}l@{}}Category Manager, Buyer, Logistic \\ Project Manager, Warehouse Manager\end{tabular}                                                                                                \\ \midrule
Switzerland      & 27                                                                                & Denner, Migros, Aldi                                                                                           & \begin{tabular}[c]{@{}l@{}}Logistic B. Denner, Product \\ Manager, Senior Group Category \\ Director, Buying Director, Country \\ Managing Director, Manager, \\ National Supply Chain Management\end{tabular} \\ \midrule
Italy            & 20                                                                                & \begin{tabular}[c]{@{}l@{}}Conad, Gruppo Selex, \\ Eurospin\end{tabular}                                       & \begin{tabular}[c]{@{}l@{}}Category Manager, Director Supply \\ Chain, National Category Manager, \\ Logistics Manager, Buyer, \\ Logistics Specialist\end{tabular}                                            \\ \midrule
Austria          & 14                                                                                & Rewe, Hofer                                                                                                    & \begin{tabular}[c]{@{}l@{}}Senior Category Manager, Head of \\ Shift Logistic, Procurement Manager, \\ Buying Manager, Buying Specialist, \\ Group Director at Supply \\ Chain Management\end{tabular}         \\ \midrule
Czech Republic   & 10                                                                                & Albert, Tesco                                                                                                  & \begin{tabular}[c]{@{}l@{}}Head of Supply Chain, Supply Chain \\ Manager, Director of Strategic Sourcing, \\ Buying and Merchandising Director\end{tabular}                                                    \\ \midrule
Poland           & 13                                                                                & Eurocash, Zabka                                                                                                & \begin{tabular}[c]{@{}l@{}}Supply Chain Planning Manager, \\ Senior Supply Chain Manager, \\ Logistics Controlling Manager, \\ Logistics Operation Manager\end{tabular}                                        \\ \midrule
Slovakia         & 3                                                                                 & Tesco                                                                                                          & \begin{tabular}[c]{@{}l@{}}Stock Controller Coordinator, \\ Big Data Analyst for Supply \\ Chain, Supply Chain Promotion Manager\end{tabular}                                                                  \\ \bottomrule
\end{tabular}
\caption{Number of people contacted for the survey}
\label{tab:survey-participants}
\end{table}

When we encountered challenges in gathering sufficient responses for our study on supply chain resilience, especially from experts in specialized fields, we found that a multi-faceted approach was necessary. We expanded our reach by leveraging our extended professional networks through platforms like LinkedIn. We urged our first or second circle contacts to reach out to their contacts and help us with the surveys and finally, we picked up the phone and tried to reach those individuals through company reception numbers and cold-calling professionals at their office numbers within our local country and beyond. Additionally, we utilized  social media forums and professional networks to connect directly with potential participants.  These strategies significantly enhanced our ability to collect the necessary data efficiently  enabling the accumulation of sufficient data to conduct a substantive analysis, ensuring that the qualitative research met statistical validity and reliability criteria.

\subsection{Limitations and Ethical Considerations}

We ensured the ethical treatment of all participants throughout the research process, strictly adhering to GDPR guidelines to protect personal data and privacy. In line with these regulations, we made certain that no personal data was recorded at any stage of the survey. Participants were informed from the outset that their responses would remain completely anonymous, and we explicitly stated that no identifiable information would be collected. This was done to maintain the confidentiality of the data and to ensur participants felt secure in providing honest responses. Participation in the survey was entirely voluntary, with no questions marked as mandatory. This approach was designed to respect the autonomy of the respondents, allowing them to skip any questions they were uncomfortable answering without affecting their ability to submit the survey. Additionally, we offered participants the option to engage in a follow-up interview for more in-depth discussion. This option was also presented as entirely voluntary, and participants were given the choice to provide their contact details only if they were interested in participating further. This approach allowed us to gather richer qualitative data from those who were willing, while still respecting the privacy of those who chose not to engage further. One of the key limitations of our study was the decision not to record any identifying information about the participants, in order to comply with ethical standards and data protection regulations. While this ensured the anonymity and confidentiality of the respondents, it also introduced a significant limitation: we were unable to link responses back to specific individuals, countries, companies, or job positions. This lack of granularity in the data meant that while we knew which countries, companies, and positions were targeted, we could not directly associate specific responses with these variables. Consequently, our analysis could not account for potential differences in responses based on these factors, which may have provided additional insights into the impact of the pandemic on different segments of the supply chain. Another limitation was related to the survey design itself. Initially, we developed a more extensive set of questions, which we sent to a small group of known contacts for feedback. The feedback indicated that the survey was too lengthy, particularly given that participation was voluntary and there were no incentives offered. Respondents expressed reluctance to complete a long survey, which led us to significantly shorten the questionnaire. The final survey consisted of only 10 questions, with only 7 designed for quantitative analysis and 1 for qualitative analysis. This reduction in the number of questions limited the depth of the data we could collect, particularly in terms of more focused or detailed inquiries that might have provided richer insights. The trade-off, however, was an increased likelihood of participation, which was crucial given the challenges of obtaining responses without offering incentives.

Thus, while we took significant steps to ensure the ethical treatment of participants and to protect their privacy, these measures also introduced limitations in the scope and depth of our analysis. The constraints imposed by the need for anonymity, the voluntary nature of the survey, and the decision to limit the number of questions all impacted the breadth and specificity of the data collected. It should be noted that these limitations are acknowledged and should be considered when interpreting the findings of this study.

\section{The Interviews}

In our research, following the initial survey phase, we recognized the need for a more in-depth qualitative exploration to complement the limited data obtained (will be discussed in further sections) from the survey responses. The objective of this next methodological step was to conduct personal interviews with professionals who could offer deeper insights into the functioning of supply chains during the COVID-19 pandemic and the subsequent adjustments made in response to the challenges faced. Given the lower-than-expected response rate to our survey, despite our efforts to streamline the questions and ensure their relevance and ease of completion, we decided to pursue interviews as a means to bolster our research findings. Although our survey was carefully designed to minimize respondent burden—featuring voluntary participation, non-mandatory questions, and a concise structure—we observed that the engagement was insufficient to draw comprehensive conclusions from the quantitative data alone. To address this limitation and enhance the robustness of our research, we moved forward with conducting personal interviews. The initial plan to recruit interview participants involved leveraging the same pool of respondents from our survey. As previously mentioned, we included a question in the survey asking if participants were willing to engage in a follow-up interview for a more detailed discussion. Unfortunately, none of the survey respondents expressed interest in participating further, leaving us without candidates from the survey pool. Consequently, we had to pivot our approach by once again reaching out to our professional networks, particularly through LinkedIn and other channels, to identify individuals who met the specific criteria required for the interview. It was essential that the interviewee had significant experience in the perishable goods sector and a background in supply chain management. This criterion was crucial to ensure that the insights provided during the interview were both relevant and valuable to our research. Consequently, this requirement also narrowed down people we could contact for the interview. After extensive outreach, we were able to secure a single interview with a participant whose qualifications were highly pertinent to our study. The individual in question had two years of direct experience in supply chain management at Nestlé, a leading company in the perishable goods industry. Additionally, the participant had professional experience with financial institutions such as S\&P. Academically, the interviewee held a master's degree in both finance and supply chain management, and a PhD in logistics supply chain, albeit with a focus on construction transport. At the time of the interview, the participant was employed as a Supply Chain Design Specialist at Volvo, adding a layer of credibility to the insights they provided.

% -------------------

Upon receiving agreement from the selected participant, one of the authors conducted the interview via an online Zoom meeting. Prior to the interview, the participant was fully informed of all ethical considerations, including the assurance that no personal data or personal information would be recorded or shared. The interview was conducted under the condition of anonymity, with the understanding that the participant's credentials would be disclosed to demonstrate their credibility and qualifications, but only with their explicit permission. The questions posed during the interview were carefully crafted based on the analysis of the survey data. Specifically, the questions were derived from the analysis of the multiple-choice questions using the PLS-SEM method, which will be elaborated upon in Section \ref{sec:pls-sem}, and from keyword analysis of the responses to subjective questions, as discussed in Section \ref{sec:keyword-analysis}. This ensured that the interview questions were directly relevant to the key points identified in the survey and allowed us to explore them in greater depth. In total, approximately 30 questions were prepared for the interview. However, it was understood that this list would serve as a flexible outline rather than a rigid script. The interview was designed to be conversational, allowing for the possibility of skipping certain questions or adding new ones depending on the flow of the discussion. Although no follow-up questions were pre-determined, the interview process included spontaneous follow-up questions to delve deeper into the insights provided by the participant. This flexibility was crucial in allowing the interviewer to explore the participant's responses more thoroughly and to gather more nuanced data. The questions covered a wide range of topics, from specific vulnerabilities within the supply chain to strategies for enhancing collaboration with suppliers and the role of technology in improving supply chain visibility and traceability. For example, one of the questions asked the participant to identify specific vulnerabilities within the supply chain, with a follow-up question probing whether they had encountered health-related problems in the food industry. Another question explored the flexibility and adaptability of the supply chain in responding to changes in demand or supply, with a follow-up asking for a specific example from the participant's experience.

The interview was conducted successfully, and the insights gathered provided valuable qualitative data that supplemented the quantitative findings from the survey. This data will be integrated into the overall analysis to provide a more comprehensive understanding of supply chain resilience during the pandemic. After the interview was completed, the actual list of questions and follow-up questions that were asked was compiled and can be found in Appendix \ref{appendix:interview}. This documentation ensures transparency in our methodology and allows for the replication of the study in future research. The interview conducted proved to be invaluable, offering detailed qualitative data that supplemented the limited quantitative results from our survey. The expertise and experience of the interviewee provided a depth of understanding that enriched our analysis, particularly regarding the resilience and adaptability of supply chains in the face of unprecedented disruptions like those caused by the COVID-19 pandemic. The insights gained from this interview will be discussed in further detail in the analysis and findings section of this paper, where we will integrate both the survey data and the qualitative insights to present a comprehensive view of the study's results.

% \todo[inline]{research process subsection to be added, where we summarise chronologically the entire process that was done}

\section{Research Process}

Given that the pandemic had ended approximately three years prior to the initiation of this study, we recognized an opportunity to gather insightful data on supply chain adaptations and learnings from this global disruption. Our research approach combined both qualitative and quantitative methods to offer a comprehensive analysis of the pandemic's impact and the subsequent strategic responses of companies. This section outlines the research process in detail, describing each step taken from the initial conceptualization to the final analysis and presentation of findings.

\subsection{Initial Conceptualization and Literature Review}

The research process began with an initial idea to explore the intersection of the COVID-19 pandemic and supply chain management, driven by the awareness that three years after the pandemic, sufficient data might be available to gain valuable insights. The focus was on understanding the specific impact on perishable goods, which are characterized by their limited shelf life and essential nature. Given the changes in consumption patterns—where consumers were reluctant to shop in person but still needed essential items like groceries—we saw an opportunity to investigate how the supply chain for perishable goods was managed during such a crisis.

To frame the study, we conducted an extensive literature review, beginning with an overview of the pandemic's effects on supply chains and moving towards a more focused examination of resilience in supply chain management. This review helped us identify gaps in the existing literature and sharpen our focus on the perishable goods sector. From this, we formulated our research questions and hypotheses, aiming to understand both the immediate impacts of COVID-19 on supply chains and the strategies employed to enhance resilience and preparedness for future disruptions.

\subsection{Development of Research Questions and Hypotheses}

Based on the insights gathered from the literature review, we identified a research gap related to the strategic responses of suppliers of perishable goods during the COVID-19 pandemic. This led to the formulation of our research questions and hypotheses, which centered around whether suppliers faced significant disruptions, the nature of these disruptions, and the extent to which new strategies were implemented to enhance resilience. The hypotheses aimed to examine both the impact of the pandemic on supply chains and the strategic measures taken by suppliers to mitigate these impacts.

%=====================
\subsection{Transition from Quantitative to Qualitative Research Approach}

Initially, the research aimed to adopt a quantitative approach to examine the impact of COVID-19 on the supply chain of perishable goods. As discussed earlier, we anticipated that sufficient data would be available to facilitate a robust quantitative analysis. Our objective was to obtain datasets that would allow us to compare variables such as sales, distribution, and buying patterns for perishable goods during the pandemic with the same variables before the pandemic. This approach was expected to provide a comprehensive understanding of the changes and disruptions experienced by supply chains in response to the pandemic. We began by searching for publicly available datasets that included sales and distribution data for perishable goods. For example, we sought data from grocery stores or companies, such as daily sales figures before, during, and after the pandemic, to analyze shifts in consumer buying behavior. Similarly, we aimed to collect data on the distribution of perishable goods to identify changes in logistics and supply chain practices during the pandemic period. However, our efforts to find complete datasets proved challenging. While we found various publicly available datasets, they often lacked critical components required for a meaningful comparative analysis. For instance, some datasets only covered periods before the pandemic, while others provided data exclusively for the post-pandemic period. Moreover, even when datasets included information for both before and after the pandemic, they frequently lacked data for the crucial period during the pandemic itself. Additionally, certain data we encountered, such as the Tesco grocery sales data, did not align with the specific timeframes we needed for analysis. These gaps and inconsistencies in the available datasets hindered our ability to construct a valid and comprehensive quantitative analysis.

After numerous attempts and extensive searches, it became evident that we could not obtain the complete datasets required to perform a reliable quantitative comparison. Recognizing the limitations of continuing with this approach, we decided to pivot towards a qualitative methodology. This shift allowed us to explore the research questions from a different angle, focusing on gaining deeper insights into the experiences and perspectives of supply chain professionals during the pandemic. At this stage, we then began the preparation of the questions for survey based on the literature review and knowledge gathered so far.

%=====================

\subsection{Data Collection and Preliminary Analysis}

The next step involved collecting data to analyze the formulated hypotheses. The data collection began with the creation of survey questions informed by our literature review. These questions included both objective (yes/no or Likert scale responses) and subjective open-ended questions to capture a range of insights from industry professionals. We distributed the interview questions to potential participants via social media, professional networks, and direct email invitations. The responses collected provided a preliminary data set for analysis. We conducted a Partial Least Squares Structural Equation Modeling (PLS-SEM) analysis to evaluate the responses to the objective questions, enabling us to identify initial patterns and relationships. Simultaneously, we conducted a keyword analysis on the subjective responses to identify frequently mentioned themes and concepts. This analysis aimed to uncover underlying insights and patterns that might not have been immediately apparent from the quantitative data alone.

\subsection{Iterative Process and Refinement of Research Approach}

Despite obtaining some initial insights, the results from the PLS-SEM analysis were not conclusive enough to confidently establish or reject our hypotheses. Recognizing the need for deeper investigation, we decided to refine our approach. We conducted a more detailed analysis of the keyword data to understand what additional information was required to address our research questions more effectively. Based on this refined understanding, we developed a new set of more detailed, subjective questions for a long-form, one-to-one interview format. This approach was intended to elicit richer, more nuanced responses and to address specific areas of ambiguity or uncertainty identified in the preliminary analysis.

\subsection{In-Depth Interview and Analysis}

The next phase involved identifying and securing a participant with substantial experience in supply chain management for perishable goods to conduct a comprehensive in-depth interview. After several attempts to reach out through professional networks, we successfully arranged an interview with a suitable candidate. The interview was conducted in a one-on-one format, allowing for a thorough exploration of the interviewee’s experiences and insights related to supply chain resilience during the pandemic. Following the completion of the in-depth interview, we conducted a qualitative analysis of the responses. This analysis was carried out using thematic coding and contextual examination, allowing us to correlate the findings with the initial quantitative data and to further explore the emergent themes. By integrating the qualitative insights with the findings from the PLS-SEM analysis and the keyword analysis, we developed a more comprehensive understanding of the strategies implemented by suppliers during the pandemic.

\subsection{Presentation of Research Findings}

Finally, the research findings were compiled and presented in the subsequent sections of this thesis. Each stage of the research process contributed to a deeper understanding of the impacts of COVID-19 on supply chain resilience for perishable goods and the strategic responses implemented by suppliers. Further details of each step, including the methods, analyses, and findings, are elaborated in the following sections.

