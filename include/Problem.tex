\chapter{Problem Formulation}

The COVID-19 pandemic has dramatically disrupted global supply chains, bringing to light the vulnerabilities and limitations inherent in traditional supply chain models, particularly for suppliers of perishable goods. These suppliers faced a unique set of challenges due to the nature of their products, which are often characterized by a short shelf life, stringent storage conditions, and susceptibility to rapid spoilage. As transportation networks were disrupted, border controls were tightened, and labor shortages occurred, the supply chains for perishable goods experienced significant strains. The initial phase of the pandemic saw unprecedented challenges such as delays in transportation, abrupt supply shortages, fluctuations in consumer demand, and logistical bottlenecks that severely impacted the efficiency and sustainability of supply chains. This research aims to thoroughly investigate the extent of these impacts, focusing on how the pandemic affected the supply chains of companies dealing with perishable goods, examining whether these disruptions resulted in operational, financial, or market-related setbacks, and analyzing the immediate and long-term responses of these companies.

The core objective of this study is to understand not only the initial impact but also the strategic measures that suppliers of perishable goods have taken to mitigate these disruptions and maintain supply chain continuity. The research seeks to determine whether companies implemented ad hoc measures to address the immediate disruptions or if they had pre-existing contingency plans that they activated in response to the pandemic. We aim to explore the range of strategies adopted, such as diversification of supply sources to reduce dependency on specific regions or suppliers, enhancements in cold chain logistics to ensure that perishable goods are stored and transported under optimal conditions, and the adoption of technology-driven solutions like digital platforms for inventory management, demand forecasting, and real-time monitoring of supply chain activities. The study also aims to evaluate the effectiveness of these strategies in minimizing the negative impacts of the pandemic and maintaining a stable flow of goods from suppliers to consumers. 

Furthermore, this research intends to explore the preparedness of these companies for future disruptions. The COVID-19 pandemic has acted as a catalyst for many businesses, prompting a re-evaluation of their supply chain strategies and fostering a shift towards more resilient and adaptive approaches. This study will investigate whether suppliers of perishable goods have proactively developed long-term plans and invested in new technologies or practices to enhance their resilience against future pandemics or other similar large-scale disruptions. The research will look into various aspects such as whether companies have established more robust partnerships with alternative suppliers, incorporated flexibility in their logistical operations, or leveraged data analytics and artificial intelligence to predict and manage risks better. It will also examine whether these companies have integrated sustainable practices to ensure that their supply chains can withstand both health-related crises and other disruptions, such as environmental or geopolitical events. Based on the aforementioned rationale we attempt to create a research question that encompasses the entire research.


\section{Research Question}

\textbf{How have suppliers of perishable goods been impacted by the COVID-19 pandemic, and what strategies have they adopted to enhance supply chain resilience during the pandemic and in anticipation of future disruptions?}

This research question is designed to capture a comprehensive understanding of the challenges and responses of suppliers of perishable goods during the COVID-19 pandemic. It seeks to assess the initial impacts on their supply chains, the specific measures taken to address these challenges, and the proactive strategies developed to bolster supply chain resilience both during the pandemic and in preparation for future disruptions. The study will explore diverse strategies, including the diversification of supply sources, improvements in cold chain logistics, and the adoption of technology-driven solutions, to mitigate risks and ensure the stability and efficiency of the supply chain in the face of ongoing uncertainties.


\section{Hypotheses Development}

Given the complexity and scope of the research question, it is essential to develop specific hypotheses that guide the investigation and provide a structured approach to understanding the impacts and strategies involved. The following hypotheses have been formulated to achieve these objectives:

\begin{itemize}
    \item \textbf{H1:} Suppliers of perishable goods faced significant disruptions in their supply chains due to the COVID-19 pandemic. This hypothesis aims to establish whether the pandemic caused substantial challenges, such as delays, shortages, or increased costs, that affected the normal functioning of supply chains for perishable goods.
    
    \item \textbf{H2:} Suppliers of perishable goods did not experience substantial disruptions in their supply chains during the COVID-19 pandemic. This hypothesis serves as a counterpoint to H1, allowing for the examination of cases where companies might have been unaffected or experienced minimal disruption, potentially due to pre-existing resilience measures.

    \item \textbf{H3:} Suppliers of perishable goods have implemented new strategies and procedures to enhance supply chain resilience and preparedness for future pandemics or similar disruptive events. This hypothesis seeks to explore whether companies have proactively adapted by adopting new strategies, such as diversifying suppliers, improving logistics and inventory management, or investing in technology.

    \item \textbf{H4:} Suppliers of perishable goods have not implemented significant changes or strategies to enhance supply chain resilience in preparation for future pandemics or similar disruptive events. This hypothesis allows for the investigation of cases where companies may not have taken substantial steps towards future preparedness, possibly due to resource constraints, uncertainty, or other factors.
\end{itemize}

These hypotheses will guide the research by focusing on both the immediate impacts of the pandemic on supply chains and the strategies adopted by suppliers of perishable goods to address these impacts and prepare for future disruptions. By examining these hypotheses, the study will provide a nuanced understanding of the varied responses of companies, identifying best practices and areas for improvement.

\paragraph*{\textbf{Note on Hypothesis Coverage:}} The hypotheses are designed to cover both the impact of the pandemic on supply chains and the subsequent strategies adopted by suppliers. While the specific strategies implemented are not detailed within the hypotheses themselves, the analysis of these strategies forms a critical part of the research.
