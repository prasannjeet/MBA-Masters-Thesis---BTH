\chapter{Literature Review}
\label{chap:literature}

The COVID-19 pandemic has profoundly disrupted global supply chains and accelerated the shift towards e-commerce, revealing both vulnerabilities and opportunities within these interconnected systems. The unprecedented scale and duration of the pandemic-induced disruptions have forced businesses worldwide to reevaluate and adapt their supply chain strategies, particularly in the context of handling perishable goods. This chapter provides an overview of the pandemic's impact on global supply chains and e-commerce, setting the stage for a deeper exploration of the concept of supply chain resilience.

Resilient supply chains have emerged as a critical focus for businesses aiming to navigate the complexities of the current and future market landscapes. Throughout this chapter we delve into the strategies for developing resilient supply chains, highlighting key approaches such as diversification of suppliers, technological advancements, and flexible logistics. These strategies are essential for mitigating risks and ensuring the continuity of supply chain operations under various disruptive conditions.

Following this, we explore existing research and identify gaps that need to be addressed to further enhance supply chain resilience. In particular, we discusse the specific challenges and strategies associated with adapting supply chain strategies for perishable goods, a sector that has faced unique difficulties during the pandemic.

And finally, we concludes by summarizing the key insights and emphasizing the importance of ongoing research to fill the identified gaps. Understanding and implementing effective supply chain strategies are imperative for businesses seeking to thrive in an increasingly unpredictable global environment.

\section{Overview of the Pandemic's Impact on Global Supply Chains and E-commerce}

The COVID-19 pandemic has ushered in an era of unprecedented challenges and transformations across the globe, significantly impacting global supply chains and e-commerce. This section delves into the multifaceted effects of the pandemic, drawing on insights from some pivotal studies that explore the resilience of supply chains, and the accelerated adoption of e-commerce.

The pandemic also significantly disrupted global supply chains, exposing vulnerabilities and prompting a reevaluation of supply chain resilience across various sectors. The surge in global e-commerce sales, which reached \$$4.9$ trillion in 2019 — an 11\% increase from the previous year—highlighted the growing demand for robust and flexible supply chains, particularly in regions like the United States, Japan, and China \parencite{KofiMensah2021Cross-BorderReview}. As physical stores closed and social distancing measures were implemented, the reliance on e-commerce intensified, pressuring supply chains to adapt quickly to meet the sudden shift in consumer demand, especially in countries like China, India, and Vietnam \parencite{KofiMensah2021Cross-BorderReview}. In Canada, the pandemic's impact on supply chains was particularly evident in the grocery sector, where the increased demand for online food procurement tested the resilience of existing supply networks. Before the pandemic, 46.4\% of consumers had not ordered food online; however, the need for contactless shopping due to health concerns caused a dramatic shift, compelling supply chains to adjust to the new demands of online food ordering across multiple generations \parencite{Charlebois2021SupplyStudy}. 

This widespread shift stressed the critical importance of strengthening supply chains to ensure they can withstand future disruptions and continue to meet consumer needs effectively. Furthermore, significant drop was observed in the volume of containers transported from China to the U.S. and the indefinite closure of factories and warehouses \parencite{Miljenovic2022PandemicsPandemic}. Despite these challenges, the pandemic highlighted the advantages of the drop shipping model, which allows for direct delivery from the manufacturer to the retailer or customer, showcasing its adaptability and efficiency in a crisis \parencite{Miljenovic2022PandemicsPandemic}. This model facilitated global delivery solutions, emphasizing the need for businesses to adopt more flexible and resilient supply chain strategies. The adoption of artificial intelligence (AI) and other technological advancements played a crucial role in combating the pandemic's challenges, including monitoring infected individuals and ensuring the delivery of food and medication \parencite{KofiMensah2021Cross-BorderReview}. 

However, the increase in online sales also raised sustainability concerns, particularly regarding increased plastic pollution and the need for sustainable practices in the e-commerce era \parencite{Charlebois2021SupplyStudy}. Additionally, the pandemic has undeniably reshaped the landscape of global supply chains and e-commerce, accelerating digital transformation and highlighting the need for resilience and adaptability in the face of global disruptions. The insights from \textcite{Charlebois2021SupplyStudy, Miljenovic2022PandemicsPandemic, Din2022ThePurchasing, KofiMensah2021Cross-BorderReview} provide a comprehensive understanding of the pandemic's impact, offering valuable lessons for businesses, policymakers, and consumers as they navigate the post-pandemic world. The transition towards e-commerce, the challenges faced by supply chains, and the opportunities for innovation and sustainability are critical areas for future research and strategic planning.

\section{Resilient Supply Chains}

The COVID-19 pandemic has underlined the critical importance of resilient supply chains in maintaining the flow of goods and services during global disruptions. This section explores the concept of supply chain resilience, pre-pandemic vulnerabilities, and strategies for developing resilient supply chains, drawing on insights from prominent studies. Supply chain resilience refers to the ability of a supply chain to anticipate, prepare for, respond to, and recover from unexpected disruptions. Resilience involves not just surviving disruptions but also thriving in the face of them by adapting and transforming supply chain operations \parencite{Mishra2024RedefiningFactors, Michel2023DimensionsPandemic, Cherrafi2022DigitalEra}. The COVID-19 pandemic has highlighted the need for resilience in the face of both internal and external systemic threats, including epidemics, pandemics, and other disruptions like natural disasters or terrorist attacks \parencite{Michel2023DimensionsPandemic}. The pandemic revealed several pre-existing vulnerabilities in global supply chains, including over-reliance on single sources of supply, lack of visibility and agility, and insufficient collaboration among supply chain partners. Over 94\% of Fortune 1,000 companies experienced disruptions due to the pandemic, exposing the fragility of global supply chains to high uncertainty, long-term disruptions, and the ripple effect propagation \parencite{Cherrafi2022DigitalEra}. These vulnerabilities spotlight the necessity for supply chains to be more than just efficient; they must also be resilient.

\subsection{Strategies for Developing Resilient Supply Chains}

To effectively develop resilient supply chains, it is crucial to understand and implement a variety of strategies that address both immediate and long-term risks. In the context of the COVID-19 pandemic, supply chain professionals and organizations have had to reconsider traditional approaches and adapt to new realities. This section explores several key strategies that emerged during the pandemic, highlighting their effectiveness in enhancing supply chain resilience and sustainability.

A critical strategy identified is the diversification of supply sources, which involves reducing reliance on a single supplier or geographic region to mitigate risks associated with supply chain concentration. This need for diversification was underscored during the pandemic, when localized outbreaks and restrictions severely disrupted supply chains that were heavily dependent on specific regions. The lessons learned from these disruptions have led to a renewed focus on regionalization and the importance of establishing a more geographically diverse supplier network. This shift is aimed at reducing the dependency on distant suppliers and mitigating the risks associated with global supply chain disruptions, as emphasized in recent studies \parencite{Cherrafi2022DigitalEra}. Alongside diversification, the integration of technology and digital transformation has become a fundamental aspect of building resilient supply chains. Digital technologies, such as advanced analytics, the Internet of Things (IoT), and blockchain, enhance supply chain visibility, agility, and collaboration by improving the sensing, seizing, and transformation capabilities within supply chains \parencite{Cherrafi2022DigitalEra}. The pandemic accelerated the adoption of these technologies, highlighting their critical role in maintaining continuity and ensuring supply chain resilience during periods of disruption \parencite{Mishra2024RedefiningFactors, Cherrafi2022DigitalEra}.

In addition to diversification and digital transformation, the flexibility of logistics and distribution models has proven essential for managing volatility in both demand and supply during the pandemic. The agility provided by on-demand warehousing and multi-modal transportation options has allowed supply chains to adapt rapidly to changing conditions. This approach has been further complemented by the simplification and innovation of supply chain designs, which help to make supply chains more predictable and manageable under uncertain conditions \parencite{Mishra2024RedefiningFactors}. Another vital strategy is enhancing collaboration and information sharing among supply chain partners. Effective collaboration ensures visibility and responsiveness, which are fundamental to resilience. 

The pandemic particularly underscored the role of information systems (IS) in enabling effective communication across the supply chain, highlighting IS's transversal importance in building a resilient supply chain \parencite{Michel2023DimensionsPandemic}. Real-world examples further illustrate how these strategies have been successfully applied. For instance, Médecins Sans Frontières Logistique (MSF Log) demonstrated an ability to reorganize in response to crises through a combination of reactive measures and proactive planning. This balance of strategies illustrates the dual approach required to build effective resilience in supply chains \parencite{Michel2023DimensionsPandemic}. Additionally, empirical studies from the manufacturing sector, such as the steel industry in India, offer valuable insights into how agile supply chain enablers can be effectively implemented in practice, providing a comprehensive understanding of supply chain management under challenging conditions \parencite{Mishra2024RedefiningFactors}.

Together, these strategies form a multifaceted approach to enhancing supply chain resilience, underscoring the importance of diversity, digitalization, flexibility, and collaboration in navigating the complexities and uncertainties of the modern supply chain environment.

\section{Synthesis and Research Gaps}

The dynamics of supply chain resilience during the COVID-19 pandemic presents a critical area of study, particularly in the handling of perishable goods. This section synthesizes insights from \textcite{Li2010TheValue, RiveroGutierrez2020OmnichannelSector, Zhang2020IntegrationBOPS}, exploring how supply chains adapted to ensure the continuity and efficiency of food supply, and identifying gaps where further research is imperative.

\subsection{Adapting Supply Chain Strategies for Perishable Goods}

The onset of the pandemic required rapid adaptation of supply chain strategies to address the unique challenges of perishable goods distribution. This adaptation was crucial to maintain freshness and reduce wastage during periods of unpredictable demand and logistical disruptions. The feasibility of innovative distribution methods, such as "buy online and pick up in store" (BOPS), became increasingly relevant \parencite{Zhang2020IntegrationBOPS}. Although initially aligned with consumer convenience, the strategic value of BOPS in enhancing supply chain resilience through diversified distribution options has become evident. Such strategies reduce the dependency on traditional supply routes and help manage the perishable goods more efficiently.

In sectors where timely delivery is critical, the integration of advanced digital solutions has proven essential. For instance, in highly regulated sectors like healthcare, the adaptation of omnichannel approaches ensures operational resilience \parencite{RiveroGutierrez2020OmnichannelSector}. The application of similar strategies in the food sector, combining digital and physical distribution channels, could significantly enhance the responsiveness and flexibility of supply chains dealing with perishable items. However, the implementation of these adaptive strategies raises questions about optimal conditions and the scalability of such models. The study by \textcite{Zhang2020IntegrationBOPS} provides a mathematical model to explore these conditions, suggesting that the scale of implementation and the specific characteristics of supply chain networks significantly influence the success of adaptive strategies \parencite{Zhang2020IntegrationBOPS}. This highlights a clear research gap in understanding how scalable and flexible supply chain models can be effectively designed and implemented to cope with future disruptions.



\section{Conclusion}

This literature review examines the complex relationship between supply chain resilience and shifts in consumer behavior, particularly in response to the COVID-19 pandemic. It highlights how the pandemic accelerated the transition to online shopping and emphasized the need for resilient supply chains to support these changes. The review identifies key strategies such as supply source diversification, technological advancements, flexible logistics, and collaboration among supply chain partners as crucial for enhancing resilience. These strategies not only mitigate disruptions but also align supply chain operations with evolving consumer preferences. Additionally, the review discusses the integration of supply chain resilience with changing consumer behavior, emphasizing the importance of omnichannel strategies to bridge online and offline shopping experiences and meet consumer expectations.


% Maybe this section can be added in the last chapter - Discussion/Conclusion, etc. 
% \subsection{Implications for Future Research and Practice}
% For researchers, this review emphasized the need for further investigation into the long-term impacts of the pandemic on consumer behavior and supply chain strategies. Future research should explore the sustainability of the shift towards online shopping and how supply chains can adapt to these changes in a post-pandemic world. Additionally, there is a need for sector-specific studies to understand the unique challenges and opportunities presented by different industries in integrating supply chain resilience and consumer behavior changes. For practitioners, the findings from this review offer several recommendations. Businesses should focus on enhancing their supply chain resilience through diversification, technological adoption, and collaboration. Embracing omnichannel strategies and investing in digital transformation are critical for meeting the evolving expectations of consumers. Practitioners should also consider the insights from consumer behavior studies to tailor their supply chain strategies, ensuring they are responsive to changes in consumer preferences and shopping habits.


% The COVID-19 pandemic has irrevocably changed the landscape of consumer behavior and supply chain management. The interconnectedness of supply chain resilience and consumer behavior changes is evident, with each influencing the other in significant ways. As the world continues to navigate the aftermath of the pandemic, the insights from this literature review provide a valuable framework for understanding these dynamics. By focusing on the key findings and implications outlined, researchers and practitioners can contribute to the development of more resilient supply chains and adaptive business strategies, ensuring they are well-equipped to meet the challenges of the future.