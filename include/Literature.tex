\chapter{Literature Review}
\label{chap:literature}

\section{Overview of the Pandemic's Impact on Global Supply Chains and E-commerce}

The COVID-19 pandemic has ushered in an era of unprecedented challenges and transformations across the globe, significantly impacting global supply chains and e-commerce. This section delves into the multifaceted effects of the pandemic, drawing on insights from some pivotal studies that explore the resilience of supply chains, and the accelerated adoption of e-commerce.

The pandemic also significantly disrupted global supply chains, exposing vulnerabilities and prompting a reevaluation of supply chain resilience across various sectors. The surge in global e-commerce sales, which reached \$$4.9$ trillion in 2019 — an 11\% increase from the previous year—highlighted the growing demand for robust and flexible supply chains, particularly in regions like the United States, Japan, and China \parencite{KofiMensah2021Cross-BorderReview}. As physical stores closed and social distancing measures were implemented, the reliance on e-commerce intensified, pressuring supply chains to adapt quickly to meet the sudden shift in consumer demand, especially in countries like China, India, and Vietnam \parencite{KofiMensah2021Cross-BorderReview}. In Canada, the pandemic's impact on supply chains was particularly evident in the grocery sector, where the increased demand for online food procurement tested the resilience of existing supply networks. Before the pandemic, 46.4\% of consumers had not ordered food online; however, the need for contactless shopping due to health concerns caused a dramatic shift, compelling supply chains to adjust to the new demands of online food ordering across multiple generations \parencite{Charlebois2021SupplyStudy}. This widespread shift stressed the critical importance of strengthening supply chains to ensure they can withstand future disruptions and continue to meet consumer needs effectively. Furthermore, significant drop was observed in the volume of containers transported from China to the U.S. and the indefinite closure of factories and warehouses \parencite{Miljenovic2022PandemicsPandemic}. Despite these challenges, the pandemic highlighted the advantages of the drop shipping model, which allows for direct delivery from the manufacturer to the retailer or customer, showcasing its adaptability and efficiency in a crisis \parencite{Miljenovic2022PandemicsPandemic}. This model facilitated global delivery solutions, emphasizing the need for businesses to adopt more flexible and resilient supply chain strategies. The adoption of artificial intelligence (AI) and other technological advancements played a crucial role in combating the pandemic's challenges, including monitoring infected individuals and ensuring the delivery of food and medication \parencite{KofiMensah2021Cross-BorderReview}. However, the increase in online sales also raised sustainability concerns, particularly regarding increased plastic pollution and the need for sustainable practices in the e-commerce era \parencite{Charlebois2021SupplyStudy}. Additionally, the pandemic has undeniably reshaped the landscape of global supply chains and e-commerce, accelerating digital transformation and highlighting the need for resilience and adaptability in the face of global disruptions. The insights from \textcite{Charlebois2021SupplyStudy, Miljenovic2022PandemicsPandemic, Din2022ThePurchasing, KofiMensah2021Cross-BorderReview} provide a comprehensive understanding of the pandemic's impact, offering valuable lessons for businesses, policymakers, and consumers as they navigate the post-pandemic world. The transition towards e-commerce, the challenges faced by supply chains, and the opportunities for innovation and sustainability are critical areas for future research and strategic planning.

\section{Resilient Supply Chains}

The COVID-19 pandemic has underlined the critical importance of resilient supply chains in maintaining the flow of goods and services during global disruptions. This section explores the concept of supply chain resilience, pre-pandemic vulnerabilities, and strategies for developing resilient supply chains, drawing on insights from prominent studies. Supply chain resilience refers to the ability of a supply chain to anticipate, prepare for, respond to, and recover from unexpected disruptions. Resilience involves not just surviving disruptions but also thriving in the face of them by adapting and transforming supply chain operations \parencite{Mishra2024RedefiningFactors, Michel2023DimensionsPandemic, Cherrafi2022DigitalEra}. The COVID-19 pandemic has highlighted the need for resilience in the face of both internal and external systemic threats, including epidemics, pandemics, and other disruptions like natural disasters or terrorist attacks \parencite{Michel2023DimensionsPandemic}. The pandemic revealed several pre-existing vulnerabilities in global supply chains, including over-reliance on single sources of supply, lack of visibility and agility, and insufficient collaboration among supply chain partners. Over 94\% of Fortune 1,000 companies experienced disruptions due to the pandemic, exposing the fragility of global supply chains to high uncertainty, long-term disruptions, and the ripple effect propagation \parencite{Cherrafi2022DigitalEra}. These vulnerabilities spotlight the necessity for supply chains to be more than just efficient; they must also be resilient.

\subsection{Strategies for Developing Resilient Supply Chains}

\begin{itemize}
    \item \textbf{Diversification of Supply Sources: }Diversification of supply sources mitigates risks associated with supplier concentration in specific regions, which became evident when localized outbreaks and restrictions led to significant disruptions \parencite{Cherrafi2022DigitalEra}. The pandemic has prompted a reevaluation of globalized supply chains, with a call for regionalization and diversification to reduce dependency on distant suppliers and mitigate risks \parencite{Cherrafi2022DigitalEra}.

    \item \textbf{Technology and Digital Transformation: }Digital technologies and circular economy practices have emerged as dynamic capabilities crucial for enhancing supply chain resilience and sustainability. These technologies can improve sensing, seizing, and transforming capabilities within supply chains, offering tools for better visibility, agility, and collaboration \parencite{Cherrafi2022DigitalEra}. The pandemic has accelerated the adoption of digital technologies, highlighting their role in ensuring supply chain continuity and resilience \parencite{Mishra2024RedefiningFactors, Cherrafi2022DigitalEra}.

    \item \textbf{Flexible Logistics and Distribution Models: }The agility of supply chains during the pandemic was crucial for managing volatility in demand and supply. Flexible logistics and distribution models, including on-demand warehousing and multi-modal transportation, have proven essential for adapting to rapidly changing conditions. Simplification of supply chains and innovative supply chain designs are suggested to manage and mitigate risks more effectively, making the system more predictable and manageable \parencite{Mishra2024RedefiningFactors}.

    \item \textbf{Collaboration and Information Sharing: }Collaboration among supply chain partners enhances visibility and responsiveness, which are vital for resilience. The importance of information systems (IS) in facilitating collaboration and ensuring effective communication across the supply chain was particularly highlighted during the pandemic. IS's involvement in all dimensions of resilience underscores its transversal importance in achieving a resilient supply chain \parencite{Michel2023DimensionsPandemic}.

    \item \textbf{Case Studies and Examples: }Real-world examples from the studies illustrate how organizations have navigated the challenges posed by the pandemic. Médecins Sans Frontières Logistique (MSF Log) demonstrated the ability to reorganize in response to crises through both reactive measures and proactive planning, emphasizing the balance of proactive and reactive measures required for effective resilience \parencite{Michel2023DimensionsPandemic}. The manufacturing industry, particularly the steel sector in India, provides empirical insights into the application of agile supply chain enablers, offering a comprehensive understanding of supply chain management in practical settings \parencite{Mishra2024RedefiningFactors}.

\end{itemize}


\section{Synthesis and Research Gaps}

The dynamics of supply chain resilience during the COVID-19 pandemic presents a critical area of study, particularly in the handling of perishable goods. This section synthesizes insights from \textcite{Li2010TheValue, RiveroGutierrez2020OmnichannelSector, Zhang2020IntegrationBOPS}, exploring how supply chains adapted to ensure the continuity and efficiency of food supply, and identifying gaps where further research is imperative.

\subsection{Adapting Supply Chain Strategies for Perishable Goods}

The onset of the pandemic required rapid adaptation of supply chain strategies to address the unique challenges of perishable goods distribution. This adaptation was crucial to maintain freshness and reduce wastage during periods of unpredictable demand and logistical disruptions. The feasibility of innovative distribution methods, such as "buy online and pick up in store" (BOPS), became increasingly relevant \parencite{Zhang2020IntegrationBOPS}. Although initially aligned with consumer convenience, the strategic value of BOPS in enhancing supply chain resilience through diversified distribution options has become evident. Such strategies reduce the dependency on traditional supply routes and help manage the perishable goods more efficiently.

In sectors where timely delivery is critical, the integration of advanced digital solutions has proven essential. For instance, in highly regulated sectors like healthcare, the adaptation of omnichannel approaches ensures operational resilience \parencite{RiveroGutierrez2020OmnichannelSector}. The application of similar strategies in the food sector, combining digital and physical distribution channels, could significantly enhance the responsiveness and flexibility of supply chains dealing with perishable items. However, the implementation of these adaptive strategies raises questions about optimal conditions and the scalability of such models. The study by \textcite{Zhang2020IntegrationBOPS} provides a mathematical model to explore these conditions, suggesting that the scale of implementation and the specific characteristics of supply chain networks significantly influence the success of adaptive strategies \parencite{Zhang2020IntegrationBOPS}. This highlights a clear research gap in understanding how scalable and flexible supply chain models can be effectively designed and implemented to cope with future disruptions.

\subsection{Identifying Research Gaps}

The studies reviewed thus far provide a foundational understanding of how supply chains have adapted during the COVID-19 pandemic, particularly in handling perishable goods. However, these insights reveal several areas where further investigation is crucial. There remains a significant gap in understanding the full impact of technological advancements on supply chain resilience. Emerging technologies such as artificial intelligence (AI) and blockchain have shown potential to revolutionize supply chain management by enhancing efficiency, transparency, and responsiveness. Detailed research is needed to evaluate how these technologies can be specifically leveraged to bolster the resilience of supply chains managing perishable goods. This is critical for ensuring that food supplies remain safe and efficient, minimizing waste and improving distribution even under disruptive conditions. Additionally, the adaptations and strategies developed for specific sectors during the pandemic, such as the implementation of advanced digital solutions in regulated environments like healthcare, suggest that similar strategies could be applied to the perishable goods sector. More focused studies are required to explore how these strategies can be tailored to meet the unique demands and challenges of different industries, ensuring that supply chains can withstand not only current disruptions but also future crises.

Furthermore, while the integration of online and offline channels in retail has been examined, primarily through models like 'buy online and pick up in store' (BOPS), there is a need to delve deeper into the scalability and practical implementation of such models across various supply chain networks. The mathematical models provided by other studies have started to address these issues, but more comprehensive research is necessary to establish best practices and identify the optimal conditions under which these integrated strategies enhance supply chain resilience. This includes investigating the sector-specific challenges and opportunities that arise when adapting supply chain designs to efficiently manage perishable goods. Each sector presents unique conditions and regulatory frameworks that can affect the implementation of resilient supply chain strategies. Therefore, research aimed at uncovering these nuances and developing sector-specific adaptations could provide invaluable guidance for businesses seeking to enhance their operational resilience in the face of ongoing and future disruptions.

\section{Conclusion}

This literature review has explored the multifaceted relationship between supply chain resilience and consumer behavior changes, particularly in the context of the COVID-19 pandemic. The review has highlighted how the pandemic has acted as a catalyst for significant shifts in consumer behavior towards online shopping and underscored the importance of resilient supply chains in supporting these changes. Through the examination of various studies, several key findings and implications for future research and practice have emerged. The review began with an overview of the impact of the pandemic on global supply chains and e-commerce, establishing a baseline for understanding the subsequent shifts in supply chain strategies. It was observed that the pandemic accelerated pre-existing trends towards online shopping, with significant increases in e-commerce adoption across diverse demographic groups and product categories. This shift necessitated a reevaluation of supply chain strategies to ensure resilience against such unprecedented disruptions. Further analysis revealed that resilient supply chains are crucial for supporting the sustained shift towards online shopping. Strategies such as diversification of supply sources, technological advancements, flexible logistics, and collaboration among supply chain partners were identified as key enablers of resilience. These strategies not only help in mitigating the impact of disruptions but also in aligning supply chain operations with changing consumer preferences. The integration of supply chain resilience and consumer behavior changes was discussed, highlighting synergies and potential conflicts. The review pointed out the importance of omnichannel strategies in bridging the gap between online and offline shopping experiences, thereby enhancing supply chain resilience and meeting consumer expectations.


% Maybe this section can be added in the last chapter - Discussion/Conclusion, etc. 
% \subsection{Implications for Future Research and Practice}
% For researchers, this review emphasized the need for further investigation into the long-term impacts of the pandemic on consumer behavior and supply chain strategies. Future research should explore the sustainability of the shift towards online shopping and how supply chains can adapt to these changes in a post-pandemic world. Additionally, there is a need for sector-specific studies to understand the unique challenges and opportunities presented by different industries in integrating supply chain resilience and consumer behavior changes. For practitioners, the findings from this review offer several recommendations. Businesses should focus on enhancing their supply chain resilience through diversification, technological adoption, and collaboration. Embracing omnichannel strategies and investing in digital transformation are critical for meeting the evolving expectations of consumers. Practitioners should also consider the insights from consumer behavior studies to tailor their supply chain strategies, ensuring they are responsive to changes in consumer preferences and shopping habits.


% The COVID-19 pandemic has irrevocably changed the landscape of consumer behavior and supply chain management. The interconnectedness of supply chain resilience and consumer behavior changes is evident, with each influencing the other in significant ways. As the world continues to navigate the aftermath of the pandemic, the insights from this literature review provide a valuable framework for understanding these dynamics. By focusing on the key findings and implications outlined, researchers and practitioners can contribute to the development of more resilient supply chains and adaptive business strategies, ensuring they are well-equipped to meet the challenges of the future.