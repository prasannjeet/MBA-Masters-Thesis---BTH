\chapter{Results}

\section{Hypothesis Evaluation}
\todo[inline]{
    Note that this is only the results based on the interview, which is also currently partial. This section will be updated with the entire results for the research in the next submission.
}

\subsection{Evaluation of Hypothesis 1 and 2}

Based on the detailed analysis of the interview, we can determine that the evidence strongly supports Hypothesis 1, which asserts that suppliers of perishable goods faced significant disruptions in their supply chains due to the COVID-19 pandemic. The interview highlighted multiple challenges, such as supply shortages, transportation delays, and the need for rapid adjustments in inventory strategies. These challenges are consistent with the notion that the pandemic caused substantial disruptions across various supply chain components. For instance, the analysis accentuates that external events, such as the blockage of the Suez Canal, created severe transportation delays, leading to bottlenecks and increased complexity in demand forecasting. Such incidents illustrate the vulnerability of supply chains to sudden disruptions and emphasize the importance of flexibility in managing these challenges. The interview further revealed that managing transportation delays and shortages necessitated frequent recalibration of logistics and inventory management strategies, underscoring the reality of significant disruptions. Moreover, the presence of labor constraints and increased costs due to global health and safety regulations further validates Hypothesis 1. Although the interview did not directly delve into labor issues, it implied that disruptions in material availability and workforce management were critical challenges, contributing to higher operational costs and supply chain instability. Conversely, there is little to no evidence supporting Hypothesis 2, which suggests that suppliers of perishable goods did not experience substantial disruptions. The cumulative evidence points towards a broad range of disruptions, affecting both upstream and downstream supply chain activities.

Thus, the evaluation concludes that Hypothesis 1 is validated by the evidence presented in the interview, while Hypothesis 2 is contradicted by the same evidence. The interviewee's account of the supply chain challenges faced during the pandemic provides clear confirmation that the disruptions were significant, widespread, and required substantial adjustments to supply chain strategies and practices.

\subsection{Evaluation of Hypothesis 3 and 4}

In evaluating Hypothesis 3, which suggests that suppliers of perishable goods have implemented new strategies and procedures to enhance resilience and preparedness, the evidence from the interview provides strong support. The interview highlighted various immediate and long-term strategies that companies adopted in response to the pandemic, such as local warehousing, diversification of suppliers, and increased collaboration with both large and small suppliers. These strategies are indicative of a proactive approach aimed at strengthening supply chain resilience and preparing for future disruptions. Further evidence supporting Hypothesis 3 is seen in the long-term strategic changes implemented by companies, such as the adoption of technology for real-time data sharing and enhanced visibility across the supply chain. The creation of crisis management teams and the development of new supplier partnerships also demonstrate a commitment to improving preparedness for future crises. These changes reflect a strategic shift towards more resilient and flexible supply chains, which aligns with the hypothesis that companies have made significant efforts to enhance their resilience. Conversely, the evidence contradicts Hypothesis 4, which posits that suppliers have not implemented significant changes or strategies to enhance supply chain resilience. The findings show that companies have actively pursued various strategic responses to address the challenges posed by the pandemic. The use of technologies, establishment of crisis management teams, and the emphasis on continuous evaluation of supplier networks all point towards a deliberate effort to improve supply chain robustness.

In summary, the evidence from the interview strongly supports Hypothesis 3, demonstrating that suppliers have implemented various strategies to enhance resilience and preparedness. Meanwhile, Hypothesis 4 is contradicted by the findings, as companies have indeed made significant strategic changes to improve their supply chain resilience in anticipation of future disruptions.

