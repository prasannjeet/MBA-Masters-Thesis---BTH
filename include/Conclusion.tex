\chapter{Conclusion}

While this thesis aimed to investigate the impact of the COVID-19 pandemic on the supply chains of perishable goods and to explore the strategies adopted by suppliers to enhance resilience during the pandemic and in anticipation of future disruptions, this study has several limitations.

 

 \section{Limitations}
In this section we discuss the challenges and obstacles we faced while performing the research. The work undertaken could be a part of a larger study and multifaceted approach to the segment dividing the perishable goods into it's subcategories such as vegetables, fruits, meat and dairy, etc.

 \subsection{Anonymity and Lack of Data Granularity}

 One of the key limitations of our study was the decision not to record any identifying information about the participants, in order to comply with ethical standards and data protection regulations. While this ensured the anonymity and confidentiality of the respondents, it also introduced a significant limitation: we were unable to link responses back to specific individuals, countries, companies, or job positions. This lack of granularity in the data meant that while we knew which countries, companies, and positions were targeted, we could not directly associate specific responses with these variables. Consequently, our analysis could not account for potential differences in responses based on these factors, which may have provided additional insights into the impact of the pandemic on different segments of the supply chain. 

 

 \subsection{Survey Design Constraints}

 Another limitation was related to the survey design itself. Initially, we developed a more extensive set of questions, which we sent to a small group of known contacts for feedback. The feedback indicated that the survey was too lengthy, particularly given that participation was voluntary and no incentives were offered. Respondents expressed reluctance to complete a long survey, which led us to significantly shorten the questionnaire. The final survey consisted of only 10 questions, with 7 designed for quantitative analysis and 1 for qualitative analysis. This reduction in the number of questions limited the depth of the data we could collect, particularly in terms of more focused or detailed inquiries that might have provided richer insights. 

The trade-off, however, was an increased likelihood of participation, which was crucial given the challenges of obtaining responses without offering incentives. While we took significant steps to ensure the ethical treatment of participants and to protect their privacy, these measures also introduced limitations in the scope and depth of our analysis. The constraints imposed by the need for anonymity, the voluntary nature of the survey, and the decision to limit the number of questions all impacted the breadth and specificity of the data collected. These limitations are acknowledged and should be carefully considered when interpreting the findings of this study.

 

 \subsection{Model Fit in PLS-SEM Analysis}
 The PLS-SEM model employed exhibited poor fit, indicating that the chosen constructs may not have fully captured the complexity of the relationships between pandemic disruptions, resilience strategies, and change management. Future research should consider additional variables that may better explain these relationships.

 

 \subsection{Sector Focus}
 The study's scope was limited to suppliers of perishable goods. While this provided valuable insights specific to this sector, the generalizability of the findings to other sectors remains limited.

 

 \subsection{Incorporating Additional Variables}
 Future studies should explore other factors influencing change management and resilience, such as organizational culture, government policies, and external market conditions. These could provide a more holistic view of the factors shaping supply chain resilience during crises.

 \section{Future Research Directions}

This study opens several avenues for future research:

\subsection{Sector-Specific Strategies}
While the findings provide a broad view of resilience strategies, more focused studies are required to explore how these strategies can be tailored to meet the unique demands and challenges of different industries. Each sector presents its own regulatory frameworks, operational dynamics, and risks, and future research should aim to develop sector-specific adaptations of resilience strategies.

\subsection{Scalability of Integrated Supply Chain Models}
While strategies such as the integration of online and offline retail channels (e.g., "buy online, pick up in-store" or BOPS) have been explored in some models, there is a need for deeper research into the scalability and practical implementation of such models across various supply chain networks. Investigating sector-specific challenges, particularly for perishable goods, could yield valuable insights into how such models enhance resilience.

\subsection{Longitudinal Studies on Strategy Effectiveness}
Future research should consider conducting longitudinal studies to track the effectiveness of the resilience strategies adopted during the pandemic over time. Understanding the long-term outcomes of these strategies could help establish best practices and identify the conditions under which these strategies succeed or need modification.

\subsection{Role of Emerging Technologies}
As digital tools and technologies play an increasing role in supply chain management, future studies should explore how technologies such as artificial intelligence, blockchain, and the Internet of Things (IoT) can be integrated into supply chain resilience strategies. A deeper exploration of their scalability, sector-specific applications, and impact on resilience would contribute significantly to the field.

 Results of our research contributes to the understanding of how suppliers of perishable goods were impacted by the COVID-19 pandemic and how they responded to enhance their supply chain resilience. The study highlights that while significant disruptions occurred, suppliers adopted a range of short-term and long-term strategies to mitigate these impacts. Diversified supplier bases, innovation, and operational continuity were among the key strategies employed, indicating a proactive approach to building supply chain resilience.

However, the limitations imposed by the need for anonymity, the shortened survey design, and the poor fit of the PLS-SEM model suggest that further research is needed to expand on these findings. More granular data collection, sector-specific studies, and the exploration of digital tools in supply chain management can provide deeper insights and improve the preparedness of supply chains for future disruptions. Future research should aim to address these gaps and help businesses enhance their operational resilience in an increasingly unpredictable global environment.

 