\chapter{Discussion}

The results of our study demonstrate significant alignment and, in some cases, divergence with the findings of previous research on supply chain resilience in the context of perishable goods during the COVID-19 pandemic. The integrated results from our PLS-SEM analysis and qualitative interviews provide insights into how suppliers were impacted by the pandemic and the strategies they adopted to enhance resilience. This discussion will compare our findings with the results reported in the studies cited above.

Our findings indicate that while there were substantial disruptions, including supply shortages, transportation delays, and increased operational costs, the suppliers did not significantly modify their management practices in response to these disruptions, as evidenced by the minimal direct impact of Pandemic Disruption on Change Management (0.002). This contrasts with the findings of \textcite{Sharma2022ImpactPerspective}, who emphasize that supplier visibility, facilitated by information sharing and supply chain traceability, significantly influences the adoption of sustainable practices and supply chain performance \parencite{Sharma2022ImpactPerspective}. The divergence can be explained by the focus of our study on the direct impact of disruption on management practices, whereas \textcite{Sharma2022ImpactPerspective}. examine how visibility affects sustainable practices indirectly. Our study highlights a more reactive approach by suppliers, primarily focusing on immediate crisis management rather than proactively enhancing sustainability. Nevertheless, it is important to note that the contrasting results obtained in our PLS-SEM analysis were also attributed to the lack of variance in our survey responses. 

With respect of another factor of Financial Performance and Preparedness, the negative impacts on financial performance due to the COVID-19 pandemic are also highlighted in our qualitative findings, where suppliers experienced increased operational costs and transportation delays. This aligns with the results of \textcite{Maharjan2023LogisticsPandemic}, who found that Japanese companies experienced both positive and negative impacts, with negative impacts predominantly affecting financial performance \parencite{Maharjan2023LogisticsPandemic}. However, our findings indicate that despite these financial setbacks, there was minimal change in management practices, whereas \textcite{Maharjan2023LogisticsPandemic} observed varying levels of preparedness and response strategies across different companies. This suggests that while the financial impact was universally felt, the extent of strategic changes and preparedness varied significantly across different contexts. 

Shifting focus to Strategies for Cold Supply Chain Resilience, both our study and the research by \textcite{Khan2023EnhancementCountry} identify the importance of logistics and inventory management in enhancing supply chain resilience. However, the authors specifically highlight "crisis simulation," "identification and securing of logistics," and "digitalization of cold supply chains" as critical strategies for maintaining the quality and resilience of cold supply chains \parencite{Khan2023EnhancementCountry}. Our findings support the adoption of these strategies, particularly the focus on logistics improvements, although "digitalization" was not identified as a key strategy, it is important to note that this was one of the strategies that was identified during the one-on-one interview discussed in the Section \ref{sec:intervew-analysis}. 

Additionally, in terms of Modeling Supply Chain Networks for Freshness and Cost Optimization, the emphasis on minimizing transmission risks and ensuring the freshness of goods in supply chains during the pandemic, as highlighted by \textcite{Asgharizadeh2023ModelingPandemic}, is consistent with our findings regarding the need for rapid adjustments to supply chain disruptions \parencite{Asgharizadeh2023ModelingPandemic}. Our study’s results, particularly from the qualitative interviews, align with the need for agile logistics strategies to maintain product freshness and manage operational costs. However, the modeling approach proposed by \textcite{Asgharizadeh2023ModelingPandemic}, involving optimization techniques and meta-heuristic algorithms, represents a more advanced and quantitative approach to addressing these challenges, which our study did not explore in depth.

When discussing the factors influencing supply chain resilience, our findings suggest that technology plays a supporting role in enhancing supply chain resilience through improved logistics and inventory management. This is partially aligned with the findings of \textcite{Ruamchart2023SupplyIndustry}, who identifies agility as the primary direct factor influencing supply chain resilience, with technology exerting an indirect effect through agility, flexibility, and collaboration \parencite{Ruamchart2023SupplyIndustry}. The indirect role of technology in our study is consistent with this view, suggesting that while technology is crucial, its primary impact may be in facilitating other resilience-enhancing factors, such as agility and flexibility. Furthermore, the concept of repurposing food supply chain management, as discussed by \textcite{Sangiumvibool-Howell2023RepurposingReview}, resonates with the strategic responses observed in our interviews, where suppliers diversified suppliers and improved logistics \parencite{Sangiumvibool-Howell2023RepurposingReview}. Both studies outline the need for adaptive strategies in managing food supply chains during the pandemic. However, our study focuses more on diversification and technological investments, while \textcite{Sangiumvibool-Howell2023RepurposingReview} emphasize re-purposing and localizing supply chains, suggesting a broader range of strategies in the literature.

In a different context of critical resilience factors, \textcite{Montanya2023ThePandemic} identify customer-oriented business awareness, proximity-based distribution models, and cooperative practices as critical resilience factors (albeit specifically in Agri-Food Supply Chains). These findings align with the strategic responses observed in our study, where diversification and improvements in logistics were key strategies adopted by suppliers. However, our study also emphasizes technological investments, which were not highlighted as prominently in \textcite{Montanya2023ThePandemic}'s review. This suggests a complementary view where both traditional (proximity-based and cooperative practices) and modern (technological) strategies are essential for building supply chain resilience.

The comparison of our findings with existing literature reveals a multidimentional approach to enhancing supply chain resilience during the COVID-19 pandemic. While there are areas of convergence, such as the emphasis on logistics improvements and the financial impacts of the pandemic, differences in the focus on technological investments and specific resilience strategies outline the diversity of approaches and contexts in supply chain management research. Our study contributes to this ongoing discourse by providing a nuanced understanding of how suppliers of perishable goods responded to the disruptions caused by the pandemic, complementing the existing body of knowledge with new insights on strategic adaptation and resilience-building efforts.

\section{Implications}

The results of this study have broader implications for both the theory and practice of supply chain management, particularly in the context of managing perishable goods during global disruptions. Beyond the immediate effects of the COVID-19 pandemic, the findings highlight the critical need for a paradigm shift in how supply chain resilience is approached. The minimal direct impact of pandemic disruptions on change management practices, as revealed by our PLS-SEM analysis, suggests that existing strategies and frameworks may be insufficient to address future crises. This calls for a more dynamic, adaptive approach that integrates flexibility, technological innovation, and robust contingency planning. Additionally, the qualitative insights from industry professionals emphasize the importance of fostering agile and responsive supply chains that can rapidly adjust to disruptions. This suggests the necessity for supply chain managers and policymakers to rethink traditional risk management strategies, moving towards a more holistic, system-wide perspective that considers both the micro and macro-level challenges in a rapidly changing global landscape. Furthermore, these results encourage further research into new models and frameworks that better capture the complexity and uncertainty of modern supply chains, ultimately contributing to a more resilient and sustainable supply chain ecosystem in the face of future global disruptions.